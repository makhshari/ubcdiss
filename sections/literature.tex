

\chapter{Related Work}
\label{ch:literature}
This chapter provides the state-of-the-art research related to our study. We discuss existing work that are relevant to bugs and challenges of IoT systems specifically, and also the existing works that follow the same empirical methodology for investigating bugs and challenges of other softwares and systems.

\textbf{Bugs and challenges of IoT systems}
Although a few previous studies have acknowledged some categories of bugs in IoT systems~\cite{IoTOSBugs,chen2017application,hnat2011hitchhiker}, no study is concerned about categorizing all types of real bugs in IoT systems using a systematic approach. 

Liang et al. \cite{IoTOSBugs} classify performance and security bugs in IoT operating systems. They inspected bug databases of the top three IoT operating systems and classify performance and security bugs into three categories based on 23 bug samples. They indicate that security or performance bugs are either abusing storage or abusing CPU time in the OS. This paper is more concerned about the hardware bugs that cause resource waste or memory failures. Based on their results, root causes of these certain types of bugs in IoT operating systems are misuse of memory-related API in hardware code, lacking memory management, code redundancy, and logic defects. This paper only focuses on bugs in IoT operating systems and their categories are more related to hardware resources and devices' constraints and thus are not representative of all types of bugs in IoT.  

Another example is a study by Zhou et al. about IoT logic bugs~\cite{zhou2019logic}. In this paper, they do a review of bugs related to the design logic of IoT systems by classifying them into seven categories and presents attack scenarios and cause analysis for each group. Their result indicates that some vulnerability bugs are inherited from traditional computing systems. While constrained resources in IoT devices and integration of diverse entities from different layers of the IoT system architecture are reasons for some other logic bugs. Similarly, this study is not generalizable to all types of IoT bugs and is more specific to IoT security flaws.

In another study~\cite{hnat2011hitchhiker}, Hnat et al. deployed over 350 sensors across 20 homes for over 8 months and observed a variety of failures. One example of such failures is process failures, due to plug disconnections, hardware failures, and driver crashes. Another failure they observed is link loss between sensors and hubs which are mostly because of concrete slab flooring, copper siding, and radio interference. Furthermore, sensor failures were observed caused by battery drain and plug disconnections. In general, a process downtime of up to 14\% was observed during their experiments, whereas sensor-process link loss and sensor failures occurred for 1-2\% of the time.

 In a recent 2020 study~\cite{IoTOSS2020}, certain peculiarities of open-source IoT repositories were analyzed via examining how developers contribute to IoT repositories. This study compares the development of 30 IoT and non-IoT repositories via factors like programming languages, specialization of contributors, the evolution of files, and the number of dependencies they have. They conclude that software development in IoT is different and it requires a diverse set of skills in various areas. However, this study does not consider bugs and experiences of IoT developers to reach conclusions about the characteristics of IoT development.

A growing body of literature has investigated issues and design flaws that cause safety and security violations in IoT systems as well as security challenges in IoT~\cite{IoTSecChallenges2017,edgeChallengesSurvey2019}. More specifically, in smart home ecosystem, security bugs related to the device firmware~\cite{nestFirmwareSecu,firmwareSec2017,firmwareSec2014}, communication protocols~\cite{protocolSec2017,protsec2016,fouladi2013honey}, smart apps, and safety of their interactions~\cite{celik2019iotguard,celik2018soteria,ISSTA2020Interactions}, as well as interactions of different components of IoT systems~\cite{securityUsenix2019} have been studied. There exist taxonomies for describing characteristics of IoT systems with respect to security and privacy concerns~\cite{alqassem2014taxonomy} \cite{chen2018internet}. However, these papers do not present their taxonomy construction process, and they are focused on a different goal, namely security requirements and attacks.

Also several studies have investigated challenges of testing IoT systems~\cite{ahmadMDBtesting2016,rosenkranz2015Testing,testingtools2018,voas2018testing}. Various solutions for IoT testing have been proposed based on e.g., model-based testing~\cite{ahmadMDBtesting2016}, IoT mutation operators and test event generators~\cite{gutierrez2019evolutionary,gutierrez2018iot}, and testing tools~\cite{testingtools2018}. Also, tools and methodologies have been proposed to aid IoT developers in developing of IoT systems~\cite{MorinUMLforIoT2017,krishna2019iot,corno2019towards}.  Morin et al. \cite{MorinUMLforIoT2017} poses heterogeneity as a significant hurdle for developers and proposes ThingML, a modelling language for IoT that is aligned with UML in addition to a code generator \cite{harrand2016thingml}, to allow rapid development of IoT systems by abstracting the diverse technologies in IoT and help the development of distinct IoT components in a platform-independent approach. There have been some other studies toward model-driven IoT development and methodological frameworks for these systems to assist IoT developers \cite{patel2015enabling}.

The challenges of developing IoT systems have been discussed from different perspectives~\cite{stojkoska2017review,vcolakovic2018IoT,hnat2011hitchhiker}. A previous study investigated the challenges of novice IoT developers to see what development tasks are more challenging for them~\cite{corno2019challenges} and developed a tool to help the novice developers~\cite{corno2019towards}. However, no study has tried to study IoT developers' challenges systematically by interviewing and surveying IoT practitioners in the field.


\textbf{Bugs and challenges of cyber-physical and embedded systems}: Chowdhury et al. characterize bugs in Cyber-Physical Systems (CPS) tool chains and propose tools for testing these types of software~\cite{chowdhury2016cyfuzz,chowdhury2018automatically}. Garousi et al. conduct a systematic literature review on testing embedded software and discuss challenges of testing embedded systems~\cite{garousi2018testing}. Some challenges mentioned in this study are the existence of non-human interfaces for manual user-interface testing, and the need for emulating electrical signal patterns for testing the behavior of the embedded system. They also study how embedded software is different from conventional software. Additionally, Arafa et al. study the behavior of developers when debugging embedded software through an exploratory study~\cite{arafa2017debugging}. Their observations suggest that locating the fault that leads to the bug, is rather harder than fixing the bug. Another interesting observation from their pilot study is that successful debuggers are more likely to consider multiple causes for a bug. Moreover, Eichelberger et al. propose a technique to reproduce bugs in embedded systems~\cite{eichelberger2017automated}. They rely on the bug classification proposed by~\cite{bugCharOpenSoftware} as their ground truth for bug categories. The classification proposed by~\cite{bugCharOpenSoftware} classifies bugs into memory bugs, concurrency bugs, and semantic bugs. However, this study only considers three large open-source projects to find the categories of bugs and thus is less likely to be generalizable for all sorts of systems. Also, this study is not focused specifically on embedded or IoT-based systems. Their subject projects includes one client-side application, one server-side application, and one operating system. Another study introduces the top five bugs in embedded software and suggests some best practices to prevent these bugs~\cite{barr2010five}. The bugs discussed in this study are race condition, non-reentrant function, missing volatile keyword, stack overflow, and heap fragmentation. Furthermore, there are some studies that focus on detecting and localizing concurrency bugs in embedded systems~\cite{asadollah2018runtime,johansson2019concurrency}. Other related studies are research on defect management in embedded systems~\cite{la2017defects}, determining application-redundant hardware faults~\cite{bartsch2017hw}, and hardware fault prediction using machine learning~\cite{khalil2020machine}. Lastly, there is sophisticated research on security-related bugs in embedded systems \cite{yuce2018fault,dessouky2019hardfails,dessouky2018patch}.

\textbf{Bugs and challenges of distributed systems}: Raposo et al. characterize faults in Wireless Sensor Networks (WSN) into the requirement, development, deployment, and operational faults~\cite{raposo2017taxonomy}. Also, Khan et al. propose a tool called Dustminer, for detecting bugs related to interactive complexity in networked sensing applications~\cite{khan2008dustminer}. In this study, they also conduct two case studies and discuss bugs and challenges in debugging two networked sensing applications. There are some existing studies on different sophisticated bugs in distributed systems. For example, Leesatapornwongsa et al. propose a taxonomy of non-deterministic concurrency bugs~\cite{leesatapornwongsa2016taxdc} in datacenter distributed systems by studying 104 distributed concurrency bugs. Other sophisticated bugs studied by previous papers are distributed timing bugs which are characterized in~\cite{li2019dfix}. In this paper, challenges of fixing these bugs are discussed, and a tool named DFix is proposed for automatically proposing patches for these bugs. Additionally, Lu et al. study the bugs related to node changes in distributed systems, which usually happen when a node joins or leaves the distributed network~\cite{lu2019understanding}. They propose tools for reproducing node change bugs and detecting null pointer exception errors, which is a common outcome of faulty node changes. Moreover, Reynolds et al. develop an infrastructure to compare the expected and actual behavior of distributed systems. Their goal is to detect structural and performance bugs in distributed systems~\cite{reynolds2006pip}. 

\textbf{Bug mining and developers' challenges}
Although mining IoT repositories has not received any attention in the literature, a growing number of studies have employed mining of software repositories or issue trackers to characterize bugs in Machine Learning systems~\cite{DlTaxFaults,zhangempiricalDL2020ICSE,islam2019comprehensive, TensorFBugsISSTA}. For instance, in a 2020 study~\cite{DlTaxFaults}, Jahangirova et al. mined DL repositories in GitHub and manually classified 271 issues and PR and 311 commits plus 477 StackOverflow discussions to classify faults in projects that use popular deep learning frameworks. They also did interviews and a validation survey, with both researchers and practitioners, to finally come up with a taxonomy of faults in using deep learning frameworks.

Additionally, \cite{islam2019comprehensive} accompanies the same goal by investigating 2716 StackOverflow posts and 500 bug fix commits in GitHub. They classify DL bug types, root causes, and impacts as well as analysis on the stage the bug happens. They also give insight toward bugs commonality and trends along the time. There are further empirical investigations conducted to study bugs in DL or ML systems by analyzing bugs in terms of their fix-time and severity\cite{MLrealBugs} \cite{FerdianMLBugs} \cite{TensorFBugsISSTA}. 


 One prior study has followed this approach to identify bug categories in Blockchain systems~\cite{blockChainBugs}. In this study, Wan et al. used mining software repositories to classify bugs in blockchain systems \cite{blockChainBugs} by manually selecting 8 blockchain repositories and classifying 1108 bug reports into different categories. Same goals have been achieved by various studies using the same methodology for Big Data computing platforms~\cite{bigDataIssues} in which authors manually inspected 210 quality issues and classified the failures and hardware faults that cause them. Other studies towards the same goal using similar approaches are in the context of web applications~\cite{jsBugsOcariza} and service compositions~\cite{chan2007fault}. 
In addition, developers' challenges have been investigated in different contexts such as mobile app development~\cite{joorabchi2013real}, and Blockchain development~\cite{zou2019smart}. 


\endinput

