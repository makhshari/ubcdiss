%% The following is a directive for TeXShop to indicate the main file
%%!TEX root = diss.tex

\chapter{Conclusions and Future Work}
\label{ch:conclusion}

In this paper, we proposed the first bug taxonomy for IoT systems. We also presented a series of categories of challenges in these systems with a qualitative study. Our findings can help both researchers and practitioners in understanding real-world pain-points of IoT development in the wild and designing new techniques and tools. Our findings shed light on the most frequent and severe IoT bugs, their correlations, and their root causes, and thereby allowing these faults to be avoided or detected early in the development of IoT systems. 

\section{Future Work}
 \textbf{Using our bug taxonomy for IoT fault seeding.}
Our bug taxonomy can be used by IoT mutation operators and test event generators as there have been recent studies regarding using these methods in IoT as well~\cite{gutierrez2019evolutionary,gutierrez2018iot}. Since we have characterized both the low-level programming errors and high-level system failures, as well as the location of the faults, our bug taxonomy is highly adaptable to seed various types of faults in IoT systems. We believe that the seeded faults using our bug taxonomy can be perfectly crafted to capture IoT developers' mistakes within various components as our taxonomy is constructed using real-world bugs in IoT projects.

Compared to the fault handling in other computing platforms such as distributed systems and cloud services, IoT systems raise unique challenges [27]. Due to the heterogeneity of IoT devices, various devices often require different fault-handling techniques executed~\cite{norris2020iotrepair}.





\endinput

